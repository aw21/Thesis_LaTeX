\section{Past Approaches to Analyzing Optimal Trade Execution}
Various approaches have been taken in the past to tackle the optimal trading execution problem. Several studies solve the optimal trade execution problem under different mathematical models of the asset price movement. \cite{A1} formulate a theoretical solution for optimal trade execution that is based on trading off execution cost mean and variance. They develop a closed form solution that maximizes expected utility of the trade at a given level of risk aversion. \cite{A1a} and \cite{A1b} expand upon this work by solving optimal trade execution under the assumption that the asset price follows a geometric Brownian motion. 

Studies have also been conducted using historical LOB data. Several studies use machine learning techniques on high-frequency data from exchanges to develop optimal trade execution methods. \cite{A5} use genetic algorithms to develop an optimal trading strategy using Australian Stock Exchange data. \cite{A4} apply reinforcement learning to optimize market making actions using equities data from across various stock market exchanges. \cite{A3} use reinforcement learning to solve the problem for several stocks listed on the NASDAQ. The studies back-test the strategies on historical data, but assume that the agent's actions do not affect the future dynamics of the LOB.

This thesis seeks to evaluate trading strategies in a realistic market environment with the impact of the agent’s actions taken into account. In order to accomplish this, an LOB must be simulated that reflects realistic market behavior. \cite{A10a} and \cite{A10b} perform statistical analysis on the empirical studies of LOBs across major exchanges. It is difficult to reflect all the properties of empirical LOBs, but studies have modelled important aspects of LOB dynamics. \cite{A11} use a supply and demand framework of LOB behavior to formulate optimal trade execution strategies involving both discrete and continuous trades. \cite{A2} expand upon this work, allowing for general LOB shapes specified density functions.

\section{LOB Queuing Model} \label{modelLOB}
Queuing models have shown promise in providing a simple model for LOB dynamics that reflect empirical properties. \cite{A6a} provides a model of the LOB where orders follow a Poisson arrival process. allowing for statistical predictions on market behaviors. \cite{A6b} model the LOB as a Markovian queuing system and estimate parameters based on data from the Tokyo stock exchange. They then use Laplace transform methods to estimate the conditional probabilities of market events based on the shape of the LOB, finding that the model provides insights on LOB behavior in the short-term. Studies have also used models similar to the queuing model where arrivals follow Poisson processes. For example, \cite{A9} modelled price jumps of an asset using a bi-dimensional Hawkes process. This model reflects several desirable properties of modern electronic markets such as its endogenous nature, where orders are correlated and made in response to each other. \cite{A6} develop a model where the LOB is a multi-dimensional queuing system centered around the mid-price of the asset. This model lends itself to easily estimating parameters from market data and accurately reproducing market behavior. This thesis uses a modified version of the model, which is described below:

The LOB is modelled as a $2K$-dimensional vector $Q$, where $K$ is number of prices on each side of the Markovian queuing system. Let $p_{ref}$ be the reference price of the of the stock. Then, $[Q_k: k = -1 \ldots -K]$ represents the number of orders on the bid side $k - 0.5$ ticks to the left of $p_{ref}$ and $[Q_k: k =1 \ldots K]$ represents the number of orders on the ask side $k - 0.5$ ticks to the right of $p_{ref}$. Then, $$Q(t) = (q_{-K}(t), … q_{-1}(t), q_1(t), … , q_K(t))$$ is a continuous-time Markov jump process with a jump size equal to one.

An event at position $k$ occurs any time that the number of shares offered at position $k$ changes. It increases when a trader places a limit order at the price of position $k$ and decreases when it is either consumed by a market order from the other side or the trader cancels the limit order.

The simplest model assumes that $Q_k$ is independent for each $k$. Each jump when $Q_k$ increases by 1 and $Q = q$ follows a Poisson arrival with rate $\lambda^+_k(q)$ and each jump when $Q_k$ decreases by 1 and $Q = q$ follows a Poisson arrival with rate $\lambda^-_k(q)$. In order to model each $Q_k$ as a positive integer with changes equal to 1, we use $Q_k$ to represent the number of Average Event Sizes ($AES_k$) present at the LOB position. $AES$ is the average size of the absolute change in the number of shares at the queue position whenever a change occurs, where each $AES_k$ could differ. In this model, $Q_k = n$ would mean that there are $n*AES_k$ shares offered at position $k$.

We choose $p_{ref}$ to be a price in between 2 adjacent ticks, so that events at each position occur at actual tick prices. Let $p_{-1}$ be the best bid price and $p_1$ be the best ask price. If $p_{-1}$ and $p_1$ are an odd number of ticks apart, then we have 
$$p_{ref}=  (p_{-1}+p_1)/2$$
If $b$ and $a$ are an even number of ticks apart, then we could pick $p_{ref}$ to be $(p_{-1}+p_1)/2 + 0.5$ or $(p_{-1}+p_1)/2 - 0.5$. We choose the one that is closest to the previous reference price.

If an event occurs that changes the best bid or ask, then we update $p_{ref}$. It can be seen that under certain conditions, this LOB model follows an ergodic Markov process. We can then use historical LOB data to estimate $AES_k$, $\lambda^+_k(q)$, and $\lambda^-_k(q)$ for each $k$.

In simulating the LOB, this thesis examines the assumptions made in this model using market exchange data. Modifications of the model are made to more accurately reflect observations from the data set, as described in Section \ref{ch:queue_model}. A market simulator is then built based on this queuing model where trade execution strategies are tested.