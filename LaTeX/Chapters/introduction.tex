\section{Problem Description} \label{ch:problemdescription}

This thesis analyzes a problem in quantitative finance known as the optimal trade execution problem. In essence, when trading a large volume of a security in a short amount of time, I seek to minimize “slippage”, which is the difference between the expected price of a security and the price at which the trade is actually executed. More formally, given a volume $V$ of a security to buy and a time limit $T$, I seek to limit the amount of money spent to acquire these shares. Similarly, we could be given an amount $V$ to sell and a time limit $T$, and seek to maximize the revenue received from selling the security. 

This problem is most relevant to a large institutional investor, such as a hedge fund, who may want to buy (respectively, sell) a large volume of a security because they believe that the price of the security will rise (fall), but find that there is not enough supply (demand) at the desired price level to satisfy the full order. If the full trade could be executed at the desired price, the hedge fund would want to execute the trade immediately in order to capture as much of the expected change in price as possible. However, the hedge fund would have to execute part of the trade at more unfavorable prices. They can instead choose to split the order over a longer time period to minimize slippage. In general, there is a trade-off between how quickly the desired position is achieved and the price impact of the trade.


\section{Background} \label{ch:background}
This problem can be illustrated more clearly by discussing the centralized limit order book (LOB). The LOB is the trading mechanism used by most exchanges around the world. The LOB for a security consists of the prices and volumes at which customers are willing to buy and sell the security (bids and asks respectively). For example, part of the LOB for Coca-Cola (KO) on the NYSE may look like Table~\ref{tab:coke1}, in which the 5 best bids and asks are listed:

\begin{table}[htbp]
\caption{Coca-Cola Limit Order Book} \label{tab:coke1}
\begin{center}
\begin{tabular}{ll|ll}
\hline \hline
\multicolumn{2}{l|}{\textbf{Bids}} & \multicolumn{2}{l}{\textbf{Asks}} \\
\hline
Volume           & Price          & Price           & Volume          \\
\hline
1000             & 48.69          & 48.70           & 500             \\
2000             & 48.68          & 48.71           & 1500            \\
3000             & 48.67          & 48.72           & 3500            \\
6000             & 48.63          & 48.75           & 2400            \\
8000             & 48.58          & 48.80           & 10000          
\end{tabular}
\end{center}
\end{table}

The price of the stock could be considered \$48.695, which is the midpoint between the best bid and best ask price. 

Say, for example, that the hedge fund receives a signal to buy 7000 shares of Coca-Cola. It could do so by issuing a market order, in which the trade is executed immediately at the best market price(s). If there is not enough supply at the best ask price, the order will progressively move up the order book until it is satisfied. In this case, it would buy 500 shares at \$48.70, 1500 shares at \$48.71, 3500 shares at \$48.72, and the remaining 1500 shares at \$48.75. It could also place a limit order, which is only executed at the specified price or better. For example, it could place a limit order for 7000 shares at \$48.67. This order would be appear in the bid side of the order book until it is matched with a market order from the ask side. The updated LOB is shown in Table~\ref{tab:coke2}.

\begin{table}[htbp]
\caption{Coca-Cola Updated Limit Order Book} \label{tab:coke2}
\begin{center}
\begin{tabular}{ll|ll}
\hline \hline
\multicolumn{2}{l|}{\textbf{Bids}} & \multicolumn{2}{l}{\textbf{Asks}} \\
\hline
Volume           & Price          & Price           & Volume          \\
\hline
1000             & 48.69          & 48.70           & 500             \\
2000             & 48.68          & 48.71           & 1500            \\
\underline{10000}             & \underline{48.67}          & 48.72           & 3500            \\
6000             & 48.63          & 48.75           & 2400            \\
8000             & 48.58          & 48.80           & 10000          
\end{tabular}
\end{center}
\end{table}

Exchanges typically institute a time priority policy, which means that orders submitted at the same price are executed in the order at which they arrived. Although the limit order has a maximum price at execution, it is not guaranteed to execute unlike the market order. 

Of course, it could also place multiple market or limit orders over time to achieve the desired position while minimizing market impact. Given a time horizon T and the state of the order book at each time step, we seek to find the optimal placement of market and limit orders to achieve the desired position. I will use the common industry benchmark of Volume Weighted Average Price (VWAP) to measure performance of our trading strategy:

$$ \frac{\sum{\text{Volume} * \text{Price}}}{\sum{\text{Volume}}}$$

In the case of market order example, the VWAP would be (500*48.70 + 1500*48.71 + 3500*48.72 + 1500*48.75)/7000 = \$48.723/share.

